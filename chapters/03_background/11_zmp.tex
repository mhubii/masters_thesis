\subsection{Zero Moment Point}
\label{sec::311_zmp}
The key metric in this work, for the generation of a dynamically balanced gait, is the zero moment point. The concept was first introduced by Miomir Vukobratovi\'{c} and Davor Juri\v{c}i\'{c} in 1968 \cite{vukobratovic1968contribution}\cite{vukobratovic1969contribution} and first utilized 1984 to generate walking trajectories for the WL-10RD robot \cite{yamaguchi1993development}. The most intuitive understanding for the ZMP arises by thinking about the realization of the simplest possible walking motion for which a humanoid robot will not fall. This motion is achieved by ensuring the feet's whole area, and not only the edge, is in contact with the ground \cite{vukobratovic2004zero}, or put in other words, we require the robot not to rotate about its feet edges. 
\\\\
Therefore, the ZMP is defined as that point on the ground at which the net moment of the inertial forces has no component along the horizontal axes \cite{hirai1998development}\cite{dasgupta1999making}. 



 
\subsubsection{Zero Moment Point of a Linear Inverted Pendulum}
\subsubsection{Measurement of the Zero Moment Point}
