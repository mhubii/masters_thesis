
\label{sec::6_pg}
The first sub-objective, which needs to be achieved, in order to let a humanoid robot maneuver an environment autonomously, is the implementation of a pattern generator that allows for the generation of dynamically balanced trajectories. The pattern generator, which got implemented within the scope of this thesis, utilizes nonlinear model predictive control, based on the works of \cite{naveau2016reactive}. The fitting of the nonlinear model predictive control into a framework that runs on Heicub bases on the ideas of \cite{stein2017closed}. However, \cite{stein2017closed} identified two main issues with previous implementations of the walking pattern generator, which will be identified and solved within this work. The general concepts, which base on the theory that got introduced in section \ref{sec::2_hw}, are, therefore, first combined into a coherent framework in section \ref{sec::61_gc}. Consequentially, details on the implementation of the framework are given in section \ref{sec::62_id}, which is followed by an exemplary usage of the implemented software in section \ref{sec::63_us}.