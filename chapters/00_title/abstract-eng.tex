In this work, we investigate the capabilities of behavioral cloning for autonomous navigation of humanoid robots from raw image input. Therefore, a nonlinear model predictive control that allows for real-time generation of stable walking trajectories is implemented and evaluated. It is demonstrated that minor modifications in the vision pipeline are sufficient for the learning of versatile motion strategies in various dynamic and static environments. This simplicity is identified as a well-suited addition to the avoidance of convex obstacles, which are represented by constraints to the solution of the implemented nonlinear model predictive control. All of the experiments are carried out on Heicub, a descendant of the iCub, which was specially designed for optimal control in locomotion at the Istituto Italiano di Tecnologia in Genova. The evaluation of balance criteria further reveals that there is no superiority of a human controller over an artificial agent. Finally, we investigate on reinforcement learning methods to replace the behavioral cloning by self-taught policies.
