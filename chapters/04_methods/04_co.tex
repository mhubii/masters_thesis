\section{Communication with Heicub}
\label{sec::34_co}

which relies on YARP \cite{metta2006yarp} to do the communication between the executing computer, and the robot, see figure \ref{fig::3_yarp}. 





the communication within the control loop is done via YARP \cite{metta2006yarp}, which works for the real robot, as well as for the simulated model in Gazebo \cite{koenig2004design} (figure \ref{fig::3_yarp}). 



It enables the user to implement control loops, like the one that got already shown in figure \ref{fig::31_pg}.

The next section - Code Structure and Usage, will therefore be mainly committed to 




\begin{figure}[h!]
	\centering
	\includegraphics[scale=.25]{chapters/04_methods/img/yarp.png}
	\caption{The pattern generation communicates to the real, or the simulated model, via YARP to access the joint motors and encoders, as well as the cameras, and the force torque sensors.}
	\label{fig::3_yarp}
\end{figure}
A part of the f 
