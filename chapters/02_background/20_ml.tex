\FloatBarrier
\section{Machine Learning}  
\label{sec::22_ml}
Machine learning methods do play a major role for autonomous navigation of robots, and whilst most recent approaches mainly dealt with tree search methods in 3D point-clouds, we aim at utilizing neural networks for solving the task at hand, since it enables us to combine spatial, semantic, and temporal understanding into one approach. Within this chapter, we will therefore explain the required fundamentals on neural networks in section \ref{sec::221_nn}, then cover two possible methods for training a neural network, one of which is supervised \ref{sec::222_bc}, whereas the second method is based on reinforcement learning \ref{sec::223_rl}, and finally explain image processing techniques in section \ref{sec::224_ip}, which allow us to extract depth maps from stereo images, so to help the neural networks understand the seen content. The goal here clearly is to introduce a method that is biologically inspired, in that it works directly in the image domain, which is very similar to how humans observe their environment. Therefore, we will shortly explain the biological similarities to a human brain within the next section - Neural Networks.