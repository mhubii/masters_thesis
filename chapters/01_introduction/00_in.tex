\label{sec::1_in}
The development of humanoid robots has ever since posed an issue of special interest to researchers. Not only is this ambition driven by human nature, which tries to understand its surroundings, including itself, but further by two additional arguments. One of which relies on the Darwinian view of biology, which regards evolution in itself as an optimization that ultimately led to two-legged motion as a global optimum. While this is true to some extent, we also need to consider that human imagination is very much bound to experience, and that there may even be more optimal solutions, which are completely suppressed by the dominance of humankind. The second argument mostly relates to the potential use of humanoid robots that is caused by artificial environmental features, which got developed by and moreover for humans. As thus, future research on humanoid robots will mainly aim at replacing human workforce from humanly designed environments including but not limited to package deliveries \cite{cassieford}, airplane construction \cite{stasse2014airbus}, applications in space, military, and for rescue missions. Ultimately, this will unlock the potential to keep humans away from hazardous encounters to avoid situations as they occurred in Chernobyl, where about 600000 \cite{kinly2006chernobyl} so-called liquidators, or simply put men, were exhibited dangerous radiation to remove radioactive material from the rooftops. Robots could have similarly been used during the Fukushima Daiichi nuclear disaster, for why the Defense Advanced Research Projects Agency (DARPA) pushed forward the development of humanoid robots. Furthermore, will they help to maintain our current living standards, which are threatened by the relative loss of workforce as a result of the demographic change. \\\\
All the above stated possible applications for humanoid robots have two things in common. To achieve them, one must investigate methods that let the robots walk through the environments without falling, and additionally one has to figure how this ability can be used to achieve certain goals while ensuring safe interaction with humans. The achievement and the combination of these aims may equivalently be regarded as the achievement of motion intelligence. Given a suitable walking pattern generator, there have been several attempts to let humanoid robots navigate the environment in a goal-driven and intelligent way. First approaches relied on methods, which were well established for wheeled robots, among them heuristic search algorithms or artificial potential fields \cite{kuffner2005motion}. Other methods introduced simultaneous localization and mapping to estimate the robot's camera position under the assumption of a linear velocity model \cite{stasse2008integrating}, which led to visual servoing that took the robot's dynamics into account \cite{dune2010cancelling}, and that got soon extended into the walking pattern generation itself to walk towards relative positions to defined objects \cite{dune2011vision}. More recent methods demonstrated impressive results, based on expensive Lidar sensors \cite{griffin2019footstep}. However, none of these methods distinguishes within the nature of objects, only the introduction of rule-based fuzzy logic \cite{dadios2012humanoid} may then trigger a variety of behaviors for different objects. All these approaches have been furthermore tailored to rather specific tasks, in that they only provide smart navigation but for example, no object manipulation. Meanwhile, the introduction of neural networks may provide a suitable solution to tackle all these unsolved issues at once, yet there has not been research on it. The contribution of this work is, therefore, to explore the use of neural networks for the achievement of motion intelligence, given inexpensive camera sensors.
\\\\
To contribute to the just stated state of the art research, a twofold objective is to be achieved within the scope of this thesis. The first sub-objective is the implementation of a nonlinear model predictive control for the generation of walking patterns, which is mainly based on the works of \cite{naveau2016reactive}, but uses a slightly modified cost function. The second objective is the design and training of a neural network architecture, which can control the pattern generation at a high level, just as a human user could. Within this thesis, the designed neural networks are trained via two different approaches. The first approach is based on behavioral cloning \cite{bojarski2016end}, while the second utilizes reinforcement learning \cite{schulman2017proximal}. To reach the objectives, a broad understanding of the underlying concepts for humanoid walking is required, which is delivered in the Theoretical Foundations for Humanoid Walking section \ref{sec::2_hw}. 



Following that, a short introduction to Heicub, the humanoid robot, which we used, is given in section \ref{sec::4_he}. 


Methods with which to implement the background, and a short note on Heicub, the humanoid robot which we used, are consequentially presented in the subsequent section \ref{sec::3_me}. Finally, experiments are carried out to evaluate the proposed methods in section \ref{sec::4_ex}, which is followed by the conclusion in section \ref{sec::7_co}. The appendix in section \ref{sec::a} then provides all details on how to install the implemented software, and how to use Heicub.
\\\\
\textbf{Note:} Best view this document in electronic form. Find a digital copy under the following link \href{https://github.com/mhubii/masters_thesis}{\underline{https://github.com/mhubii/masters\_thesis}}.
