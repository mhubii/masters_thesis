\label{sec::1_in}
The development of humanoid robots has ever since posed an issue of special interest to researchers. Not only is this ambition driven by the human's nature, which tries to understand its surroundings, including itself, but further by two additional arguments. One of which relies on the Darwinian view of biology, which states that evolution in itself may be regarded as an optimization, which ultimately led to two legged motion as a global optimum. While this is true to some extend, we also need to consider that human imagination is very much bound to experience, and that there may even be more optimal solutions, which are completely suppressed by the dominance of human kind. The second argument mostly relates to the potential use of humanoid robots that is caused by artificial environmental features, which got developed by and moreover for humans. As thus, future research on humanoid robots will mainly aim at replacing human workforce from humanly designed environments including but not limited to package deliveries \cite{cassieford}, airplane construction \cite{stasse2014airbus}, applications in space, military, and for rescue missions. Ultimately, this will unlock the potential to keep humans away from hazardous encounters to avoid situations as they occurred in Chernobyl, where about 600000 \cite{kinly2006chernobyl} so called liquidators, or simply put men, were exhibited dangerous radiation to remove radioactive material from the rooftops. Robots could have similarly been used during the Fukushima Daiichi nuclear disaster, for why the Defense Advanced Research Projects Agency (DARPA) pushed forward the development of humanoid robots. Furthermore, will they help to maintain our current living standards, which are threatened by the relative loss of workforce as a result of the demographic change. \\\\
All of the above stated possible applications for humanoid robots have two things in common. To achieve them, one has to investigate on methods that let the robots walk through the environments without falling, and additionally one has to figure how this ability can used to achieve certain goals while ensuring safe interaction with humans. The achievement and the combination of these aims may equivalently be seen as the achievement of motion intelligence. Given a suitable walking pattern generator, there have been several attempts to let humanoid robots navigate the environment in a goal-driven and intelligent way. First approaches relied on methods, which were well established for wheeled robots, among them heuristic search algorithms or artificial potential fields \cite{kuffner2005motion}. Other methods introduced simultaneous localization and mapping to estimate the robot's camera position under the assumption of a linear velocity model \cite{stasse2008integrating}, which led to visual servoing that took the robot's dynamics into account \cite{dune2010cancelling}, and that got soon extended into the walking pattern generation itself to walk towards relative positions to defined objects \cite{dune2011vision}. More recent methods demonstrated impressive results, based on expensive Lidar sensors \cite{griffin2019footstep}. However, none of these methods distinguishes within the nature of objects, only the introduction of rule based fuzzy logic \cite{dadios2012humanoid} may then trigger a variety of behaviors for a different objects. All of these approaches have been furthermore tailored to rather specific tasks, in that they only provide smart navigation but for example no object manipulation. Meanwhile, the introduction of neural networks may provide a suitable solution to tackle all of these unsolved issues at once, yet there has not been research on it. The contribution of this work is therefore to explore the use of neural networks for the achievement of motion intelligence, given inexpensive camera sensors.
\\\\
In order to contribute to the just stated state of the art research, a twofold objective is to be achieved within the scope of this thesis. The first sub-objective is the implementation of a nonlinear model predictive control for the generation of walking patterns. The second objective is the design and training of a neural network architecture, which is capable of controlling the pattern generation at a high level, just as a human user could. To reach the objectives, a broad understanding of the underlying concepts is required, which is explained in depth in the background \ref{sec::2_bg}. Methods with which the background can be implemented, and a short note on Heicub, the humanoid robot which we used, is consequentially presented in the subsequent section \ref{sec::3_me}. Finally, experiments are carried out to evaluate the proposed method in section \ref{sec::4_ex}, which is followed by the conclusion in section \ref{sec::5_co}. The appendix in section \ref{sec::a} then provides all details on how the implemented software can be installed, and how Heicub can be used.
\\\\
\textbf{Note:} This document is best viewed in electronic form. A digital copy can be found under the following link \href{https://bit.ly/2PcqX6x}{\underline{https://bit.ly/2PcqX6x}}.
