\label{sec::1_in}
The development of humanoid robots has ever since posed an issue of special interest to researchers. Not only is this ambition driven by human nature, which tries to understand its surroundings, including itself, but further by two additional arguments. The first argument regards two-legged motion, and therefore humanoid robots, as optimally. This argument relies on the Darwinian view of biology, which regards evolution in itself as an optimization that ultimately led to two-legged motion as a global optimum. While this is true to some extent, one also needs to consider that human imagination is very much bound to experience, and that there may even be more optimal walking solutions, which are completely suppressed by the dominance of humankind. The second argument mostly relates to the potential use of humanoid robots in environments that got tailored for humans, such as houses. Therefore, future research on humanoid robots will mainly aim at replacing human workforce from humanly designed environments. Humanoid robots will, therefore, for example, be useful for package deliveries \cite{cassieford} or for airplane constructions \cite{stasse2014airbus}. Commercially driven research on humanoid robots will ultimately also unlock the potential to keep humans away from hazardous encounters as they occurred in Chernobyl, where about 600000 \cite{kinly2006chernobyl} men were exhibited dangerous radiation to remove radioactive material from the rooftops. Robots could have similarly been used during the Fukushima Daiichi nuclear disaster, for why the Defense Advanced Research Projects Agency (DARPA) pushed forward the development of humanoid robots. Furthermore, will they be necessary to maintain our current living standards, which are threatened by the relative loss of workforce as a result of the demographic change. \\\\
All the above stated possible applications for humanoid robots have two things in common. To achieve them, one must investigate methods that let the robots walk through the environments without falling, and additionally, one has to figure how this ability can be used to achieve certain goals while ensuring safe interaction with humans. The achievement and the combination of these aims may equivalently be regarded as the achievement of motion intelligence. Given a suitable walking pattern generator, there have been several attempts to let humanoid robots navigate the environment in a goal-driven and intelligent way. First approaches relied on methods, which were well established for wheeled robots, among them heuristic search algorithms or artificial potential fields \cite{kuffner2005motion}. Other methods introduced simultaneous localization and mapping to estimate the robot's camera position under the assumption of a linear velocity model \cite{stasse2008integrating}, which led to visual servoing that took the robot's dynamics into account \cite{dune2010cancelling}, and that got soon extended into the walking pattern generation itself to walk towards relative positions to defined objects \cite{dune2011vision}. More recent methods demonstrated impressive results, based on expensive Lidar sensors \cite{griffin2019footstep}. However, none of these methods distinguishes within the nature of objects, only the introduction of rule-based fuzzy logic \cite{dadios2012humanoid} may then trigger a variety of behaviors for different objects. All these approaches have been furthermore tailored to rather specific tasks, in that they only provide smart navigation but for example, no object manipulation. Meanwhile, the introduction of neural networks may provide a suitable solution to tackle all these unsolved issues at once, yet there has not been research on it. The contribution of this work is, therefore, to explore the use of neural networks for the achievement of motion intelligence, given inexpensive camera sensors.
\\\\
To contribute to the just stated state of the art research, a twofold objective is to be achieved within the scope of this thesis. The first sub-objective is the implementation of a nonlinear model predictive control for the generation of walking patterns, which is mainly based on the works of \cite{naveau2016reactive}, but uses a slightly modified cost function. The second objective is the design and training of a neural network architecture, which can control the pattern generation at a high level, just as a human user could. Within this thesis, the designed neural networks are trained via two different approaches. The first approach is based on behavioral cloning \cite{bojarski2016end}, while the second utilizes reinforcement learning \cite{schulman2017proximal}. To reach the objectives, it is first required to have a broad understanding of the underlying concepts for humanoid walking, which is delivered in section \ref{sec::2_hw}. Consequentially, an extensive introduction to the required image processing, and machine learning methods for the second objective is given in the subsequent sections \ref{sec::3_ip}, and \ref{sec::4_ml}, respectively. Following that, Heicub, the humanoid robot, which is used, is described in section \ref{sec::5_he}. Details on the implementation of the walking pattern generator are given in section \ref{sec::6_pg}, which is then used to demonstrate the neural network-based high-level control in section \ref{sec::7_ah}. Furthermore, the implemented framework's application to Heicub is explained in section \ref{sec::8_co}, which finally leads to the experiments that are carried out to evaluate the proposed concepts in sections \ref{sec::9_uc} to \ref{sec::11_aw}, from where conclusions are drawn in section \ref{sec::12_co}. The appendix in section \ref{sec::a} then provides all details on how to install the implemented software, and how to use Heicub.
\\\\
\textbf{Note:} Best view this document in electronic form. Find a digital copy under the following link \href{https://github.com/mhubii/masters_thesis}{\underline{https://github.com/mhubii/masters\_thesis}}.
