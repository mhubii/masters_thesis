\label{sec::1_in}
The development of humanoid robots has ever since posed an issue of special interest to researchers. Not only is this ambition driven by the human's nature, which tries to understand its surroundings, including itself, but further by two additional arguments. One of which relies on the Darwinian view of biology, which states that evolution in itself may be regarded as an optimization, which ultimately led to two legged motion as a global optimum. While this is true to some extend, we also need to consider that human imagination is very much bound to experience, and that there may even be more optimal solutions, which are completely suppressed by the dominance of human kind. The second argument mostly relates to the potential use of humanoid robots that is caused by artificial environmental features, which got developed by and moreover for humans. As thus, future research on humanoid robots will mainly aim at replacing human workforce from humanly designed environments including but not limited to package deliveries \cite{cassieford}, airplane construction \cite{stasse2014airbus}, applications in space, military, and for rescue missions. Ultimately, this will unlock the potential to keep humans away from hazardous encounters to avoid situations as they occurred in Chernobyl, where about 600000 \cite{kinly2006chernobyl} so called liquidators, or simply put men, were exhibited dangerous radiation to remove radioactive material from the rooftops. Robots could have similarly been used during the Fukushima Daiichi nuclear disaster, for why the Defense Advanced Research Projects Agency (DARPA) pushed forward the development of humanoid robots. Furthermore, will they help to maintain our current living standards, which are threatened by the relative loss of workforce as a result of the demographic change. \\\\
All of the above stated possible applications for humanoid robots have two things in common. To achieve them, one has to investigate on methods that let the robots walk through the environments without falling, and additionally one has to figure how this ability can used to achieve certain goals while ensuring safe interaction with humans. The achievement and the combination of these aims may equivalently be seen as the achievement of motion intelligence. Given a suitable walking pattern generator, there have been several attempts to let humanoid robots navigate the environment in a goal-driven and intelligent way. First approaches relied on methods, which were well established for wheeled robots, among them heuristic search algorithms or artificial potential fields \cite{kuffner2005motion}. Other approaches relied on simultaneous localization and mapping to estimate the robot's camera position under the assumption of a linear velocity model \cite{stasse2008integrating}, which led to visual servoing under 
None of these methods however 
%All of these methods are rather tailored to specific models or rely on expensive Lidar sensors, yet there may 


% nn for nav
% in the light of tpus
\cite{jouppi2017datacenter} % google tpus 15-30x faster than gpus and cpus
\section{Aims and Objectives}
\label{sec::14_ao}
% therefore implement nmpc
% develop deep learning methods that can be used to solve the task

